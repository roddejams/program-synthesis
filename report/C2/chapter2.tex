\chapter{Background}

\section{Answer Set Programming}

\section{Inductive Functional Programming}

\begin{itemize}
\item Inductive Functional Programming is the automatic synthesis of program
\item Applications
\item Traditionally, there have been two approaches to IFP. The analytical approach performs pattern matching on the given examples, usually performing a two-step process of first generalising the examples and then folding this generalisation into a recursive program. The "generate and test" search approach works by using the examples as a basis to generate an infinite stream of candidate programs and then test these candidates to see if they correctly model the examples. \\ \\
Each approach has its own advantages and disadvantages. The analytical approach, while fast, can be very limited in its target language and the types of programs it can generate, typically being limited to reasoning about data structures such as lists or trees. On the other hand, the search based approach is a lot less restricted but has significantly worse performance due to the potentially huge search space. Through 
\item Overview of current approaches and how they relate
\end{itemize}

\pagebreak
\renewcommand\bibname{{References}}
\bibliography{References}
\bibliographystyle{plain}