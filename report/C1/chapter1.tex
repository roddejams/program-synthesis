\chapter{Introduction}

Inductive Functional Programming (IFP) is the automatic synthesis of declarative programs from an incomplete specification, typically given as pairs of Input-Output examples. Whilst this field has existed since the 1970s, recent work has slowed due to limitations on the complexity and structure of programs that can be learned. \\ \\
Answer Set Programming is a relatively new approach to logic programming, which works by computing the so called "Answer Sets" of a logic program which correspond the minimal models of that program. ASP has the advantage over traditional logic programming languages (i.e Prolog) that it is much more expressive,  \\ \\
My project introduces a new approach to IFP, through the use of ASP. Being oriented towards difficult search problems, ASP has been successfully applied to similar problems in the fields of planning, robotics and ontologies. Since one approach to IFP categorises the learning problem as a search problem over the range of possible programs, it seemed natural to apply ASP in this area. \\ \\
My desire for this project is not just to develop a new approach to IFP, but to make one that is easier to use as well. As they are developed by academics, existing IFP are either lacking user interfaces or have rudimentary ones, often making usage or translating results difficult. Through exploring ways to make the UI easier to use, it becomes possible to start considering possible application of this technology. \\ \\
I plan to focus my project as a learning tool. It is not uncommon for a new Haskell student to be unfamiliar with recursion, having no experience in writing programs in a functional style. An easy to use IFP system could help beginners by allowing them to experiment with creating different programs, helping them become more familiar with the Haskell syntax.

\pagebreak
%\addcontentsline{toc}{section}{References}
%\renewcommand\bibname{{References}}
%\bibliography{References}
%\bibliographystyle{plain}