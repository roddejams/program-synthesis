\chapter{Introduction}

Inductive Functional Programming (IFP) is the automatic synthesis of declarative programs from an incomplete specification, typically given as pairs of Input-Output examples. Whilst this field has existed since the 1970s, recent work has slowed due to limitations on the complexity and structure of programs that can be learned. \\

My project introduces a new approach to IFP, through the use of Answer Set Programming (ASP). Last year there was a successful project to inductively compute imperative programs using ASP so this project aims to apply a similar technique to functional programs, with the difference being that functional programming is focused more approaches like recursion than the imperative approach. \\

ASP is a relatively new approach to logic programming, which works by computing the so called "Answer Sets" of a logic program which correspond the minimal models of that program. ASP has the advantage over traditional logic programming languages (i.e Prolog) that it is much more expressive, which gives me a lot of flexibility in my solution.


\pagebreak
%\addcontentsline{toc}{section}{References}
\renewcommand\bibname{{References}}
\bibliography{References}
\bibliographystyle{plain}