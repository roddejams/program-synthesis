\chapter{Introduction}

Inductive Functional Programming (IFP) is the automatic synthesis of declarative programs from an incomplete specification, typically given as pairs of Input-Output examples. Whilst this field has existed since the 1970s, recent work has slowed due to limitations on the complexity and structure of programs that can be learned. \\ \\
Answer Set Programming is a relatively new approach to logic programming, which works by computing the so called "Answer Sets" of a logic program which correspond the minimal models of that program. ASP has the advantage over traditional logic programming languages (i.e Prolog) that it is much more expressive,  \\ \\
My project introduces a new approach to IFP, through the use of ASP. Why ASP? \\ \\
My desire for this project is not just to develop a new approach to IFP, but to make one that is easier to use as well. As they are developed by academics, existing IFP are either lacking user interfaces or have rudimentary ones, often making usage or translating results difficult. By making the UI easier to use, this is good. Why? \\ \\
Together with the UI, by generating
\begin{itemize}
\item Learning tool with spreadsheet
\item Initial (Basic) feature list to make reference to in conclusions
\end{itemize}

\pagebreak
%\addcontentsline{toc}{section}{References}
%\renewcommand\bibname{{References}}
%\bibliography{References}
%\bibliographystyle{plain}