\chapter{Method}

\section{A Haskell Interpreter in ASP}

As a way of better understanding how to represent the target language of the tool, my first step was to implement an Interpreter for the simple target language of my tool.

\subsection{Target Language}

Initially, I chose to target a simple sub language of Haskell with the following terms :

\begin{itemize}
\item Addition
\item Subtraction
\item Multiplication
\item Function Calls (including recursion) with any number of arguments.
\item Lists
\end{itemize}

I chose these terms as I felt they were expressive enough to be able to represent sufficiently complicated test examples.

I then represent each line of the target Haskell program with a Rule with a rule(Num, FuncName, Input, Output) predicate in the head. For example, we can represent the recursive factorial Haskell program


\section{Initial Learning}

\section{Plans}

\pagebreak
\renewcommand\bibname{{References}}
\bibliography{References}
\bibliographystyle{plain}