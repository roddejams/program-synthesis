\chapter{Method}

\section{A Haskell Interpreter in ASP}

As a way of better understanding how to represent the target language of the tool, my first step was to implement an Interpreter for the simple target language of my tool.

\subsection{Target Language}

Initially, I chose to target a simple sub language of Haskell with the following terms :

\begin{itemize}
\item Addition
\item Subtraction
\item Multiplication
\item Function Calls (including recursion) with any number of arguments.
\item Lists
\end{itemize}

I chose these terms as I felt they were expressive enough to be able to represent sufficiently complicated test examples.

I then represent each line of the target Haskell program with a Rule with a rule(Num, FuncName, Input, Output) predicate in the head. For example, we can represent the recursive factorial Haskell program

\lstinputlisting[language=Haskell, firstline=5, lastline=7]{../examples/factorial_obs.lp}

Is represented by the ASP program

\lstinputlisting[language=Prolog, firstline=11, lastline=12]{../examples/factorial_obs.lp}

We add the input atom to the body of the rule to constrain the grounding of these rules. \\

\subsection{Evaluating Rules}

The goal of my interpreter is to be able to evaluate these rules with some given input, to get the correct output. Because we are interpreting Functional Programs we can be sure each input has one output.

\lstinputlisting[language=Prolog, firstline=58, lastline=58]{../rules.lp}

This rule generates output predicates if there is a rule that best matches the given arguments, and the output of that rule has a known value.

\lstinputlisting[language=Prolog, firstline=60, lastline=65]{../rules.lp}

These rules handle calculating the value of expressions. The eval() predicate is generated whenever you need to calculate the value of an expression.

\lstinputlisting[language=Prolog, firstline=67, lastline=76]{../rules.lp}

These rules handle which expressions to calculate values for. The most important rule here is to top one, which says that whenever you have an input to some matching rule then you need to evaluate the output of that rule.

\lstinputlisting[language=Prolog, firstline=78, lastline=78]{../rules.lp}

This rule generates more input predicated if we have to calculate the value for a call to a function. This allows us to then generate more ground rule predicates and repeat.

\lstinputlisting[language=Prolog, firstline=80, lastline=81]{../rules.lp}

These rules allow us to match to the most fitting rule. A rule matches some arguments if there is no smaller rule that also matches those arguments. The match2() predicate is there as a placeholder, and will have effect when more complicated matching (e.g guards) will be needed.

\section{Initial Learning}

I decided to base my (initial) learning task as an ASP encoding similar to ASPAL, with the goal to learn the correct rule() predicates given a number of input/output example pairs:

\begin{itemize}
\item My background knowledge is simply my interpreter from 2.1.2
\item I characterised the Skeleton Rules as a choice rule where there can be up to a given number of rule predicates with an expression that is structured correctly. 
\lstinputlisting[language=Prolog, firstline=41, lastline=41]{../rules.lp}
\item I then add a constraint to remove any answer sets which do not produce the correct output against the given examples.
\lstinputlisting[language=Prolog, firstline=39, lastline=39]{../rules.lp}
\end{itemize}

\section{Future Work}

\begin{itemize}
\item Before the end of this term (~11th March) I hope to have a basic learning system in place where I can learn programs with an arbitrary number of arguments, including lists. 
\item My goals after Easter will be primarily focused on extending the language bias to deal with guards, let / where statements and higher order functions. At the same time I would like to attempt to apply existing IFP optimisations to my implementation and run experiments to compare my new tool to the most up to date IFP algorithms.
\end{itemize}

\pagebreak
%\renewcommand\bibname{{References}}
%\bibliography{References}
%\bibliographystyle{plain}